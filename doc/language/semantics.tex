% Copyright (c) 2016 Genome Research Ltd.
% 
% Author: Christopher Harrison <ch12@sanger.ac.uk>
% 
% This file is part of stag.
% 
% stag is free software: you can redistribute it and/or modify it under
% the terms of the GNU General Public License as published by the Free
% Software Foundation, either version 3 of the License, or (at your
% option) any later version.
% 
% This program is distributed in the hope that it will be useful, but
% WITHOUT ANY WARRANTY; without even the implied warranty of
% MERCHANTABILITY or FITNESS FOR A PARTICULAR PURPOSE. See the GNU
% General Public License for more details.
% 
% You should have received a copy of the GNU General Public License
% along with this program. If not, see <http://www.gnu.org/licenses/>.

\section{Semantics}

\subsection{Grouping and Aggregation}

% FIXME? This is quite unclear, or at least lacks true "formality"...

Let $S$ be an input stream, where $S_1, S_2, \ldots, S_i, \ldots$
represent each row of the stream. For every $i$, let $S_i=(C_{i,1},
C_{i,2}, \ldots, C_{i,n})$ represent the column data in that row. (Note
that if the row does not contain any split points, $C_{i,1}$ will always
exist.) Let $C_{(j)}=(C_{1,j}, C_{2,j}, \ldots)$ represent the tuple of
column data from the stream at index $j$.

For $m\geq 1$, let $O=(O_1, O_2, \ldots, O_m)$ define the output
columns. Without loss of generality, $\exists k\in\{1, \ldots, m\}$
such that, $\forall i\leq k$, $O_i$ is a function of at least $C_{(a)}$,
for some $a\in\{1, \ldots, n\}$. (Note that the aggregate functions
could potentially involve multiple column tuples.)

If $k<m$, then let the grouping tuple for each stream row $i$ be defined
as $G_i=(O_{k+1}, \ldots, O_m)$; if $k=m$, then let $G_i=i$. The
unsorted output is then given by $S$ applied to $O$, partitioned by
$\{G_i | \forall i\}$.

\subsection{Typing Discipline}

All input data comes in the form of strings. Type coercion rules apply
when operating with another type, whether defined by the function or
operator, or implied by literal. (Type casting functions also exist to
make this process explicit.)

\subsubsection{Data Types}

\paragraph{Numeric}

\paragraph{Textual}

\paragraph{Temporal}

\paragraph{Regular Expressions}

\subsubsection{Type Coercion}

\subsection{Precedence and Associativity}
