% Copyright (c) 2016 Genome Research Ltd.
% 
% Author: Christopher Harrison <ch12@sanger.ac.uk>
% 
% This file is part of stag.
% 
% stag is free software: you can redistribute it and/or modify it under
% the terms of the GNU General Public License as published by the Free
% Software Foundation, either version 3 of the License, or (at your
% option) any later version.
% 
% This program is distributed in the hope that it will be useful, but
% WITHOUT ANY WARRANTY; without even the implied warranty of
% MERCHANTABILITY or FITNESS FOR A PARTICULAR PURPOSE. See the GNU
% General Public License for more details.
% 
% You should have received a copy of the GNU General Public License
% along with this program. If not, see <http://www.gnu.org/licenses/>.

\section{Grammar}

This section illustrates the formal grammar of the \stag\ language using
syntax diagrams. Additional descriptive notes are included where clarity
is needed. By convention, arbitrary whitespace can interpose nodes and
all keywords (i.e., non-literals) are case-insensitive; wherever this
isn't the case, it will be mentioned explicitly.

Terminal nodes, where not otherwise defined, are per the ABNF
specification\cite{RFC5234}.

\subsection{\stag\ Statement}

The top-level node for all programmatic constructs is the \stag\
statement. This is presented first, but thereafter we build upwards to
ultimately arrive at definitions for each clause. The \stag\ statement
is closely modelled on an SQL \texttt{select \ldots\ group by \ldots}
statement, with the principle difference of not having a \texttt{group
by} clause of its own as it can be inferred from the scalar columns in
the output.

\begin{grammar}
  <statement> ::= \begin{syntdiag}
    <out-list>
    \begin{stack}
      <from-clause> \\
    \end{stack}
    \begin{stack}
      <split-clause> \\
    \end{stack}
    \begin{stack}
      <when-clause> \\
    \end{stack}
    \begin{stack}
      <sort-clause> \\
    \end{stack}
    \begin{stack}
      <extend-clause> \\
    \end{stack}
  \end{syntdiag}

  <comment> ::= \begin{syntdiag}
    `#'
    <to-eol>
  \end{syntdiag}
\end{grammar}

Line comments can be interposed anywhere between nodes and extend until
the end of the line on which they're placed. During parsing, they should
be reduced to a single whitespace element.

\subsection{Literals}

Literals symbolise the serialisation of data in the various types that
are representable in the \stag\ language. In all cases, whitespace is
meaningful and generally shouldn't appear between nodes.

\begin{grammar}
  <data-literal> ::= \begin{syntdiag}
    \begin{stack}
      <numeric-literal> \\
      <string-literal> \\
      <datetime-literal> \\
      <regexp-literal>
    \end{stack}
  \end{syntdiag}

  <symbol-literal> ::= \begin{syntdiag}
    \begin{stack}
      <ALPHA> \\
      `_'
    \end{stack}
    \begin{rep}
      \begin{stack}
        <ALPHA> \\
        <DIGIT> \\
        `_'
      \end{stack}
    \end{rep}
  \end{syntdiag}
\end{grammar}

\subsubsection{Numeric Literals}

Numeric literals can encode integers and floating point numbers to
arbitrary precision. Their actual value may be truncated, upon
deserialisation, according to the semantics of the numeric type [REF].

\begin{grammar}
  <integer> ::= \begin{syntdiag}
    \begin{stack}
      `-' \\
    \end{stack}
    \begin{rep}
      <DIGIT>
    \end{rep}
  \end{syntdiag}

  <number> ::= \begin{syntdiag}
    <integer>
    \begin{stack}
      `.' \begin{rep}<DIGIT>\end{rep} \\
    \end{stack}
  \end{syntdiag}

  <numeric-literal> ::= \begin{syntdiag}
    <number>
    \begin{stack}
      `e' <number> \\
    \end{stack}
  \end{syntdiag}
\end{grammar}

So called ``scientific notation'' may be used to express orders of
magnitude using the `e` infix.

\subsubsection{String Literals}

\begin{grammar}
  <string-literal> ::= \begin{syntdiag}
    <DQUOTE>
    <escaped-string>
    <DQUOTE>
  \end{syntdiag}
\end{grammar}

A string literal consists of a sequence of bytes, enclosed by double
quote characters. The byte data should be understood as UTF-8 encoded
Unicode codepoints. The data can contain special characters, such as the
double quote delimiters, by escaping them with a backslash. The
following escape sequences should be recognised:

\begin{grammar}
  <escape-sequence> ::= \begin{syntdiag}
    `\textbackslash'
    \begin{stack}
      `n' \\
      `r' \\
      `t' \\
      `\textbackslash' \\
      <DQUOTE> \\
      `u' <unicode-codepoint> `;'
    \end{stack}
  \end{syntdiag}
\end{grammar}

From top-to-bottom, these represent a newline, carriage return,
horizontal tab, backslash and double quote character, respectively.
Explicit Unicode codepoints can be serialised using the `u' escape
sequence followed by 2--6 hexadecimal digits and a terminating
semicolon.

\subsubsection{Datetime Literals}

\begin{grammar}
  <datetime-literal> ::= \begin{syntdiag}
    <DQUOTE>
    \begin{stack}
      <date-time> \\
      <full-time> \\
      <partial-time> \\
      <full-date>
    \end{stack}
    <DQUOTE>
  \end{syntdiag}
\end{grammar}

Datetime literals consist of a sequence of bytes, enclosed by double
quote characters. The byte data should be understood per the specific
rules defined in RFC3339\cite{RFC3339}.

\subsubsection{Regular Expression Literals}

\begin{grammar}
  <regex-literal> ::= \begin{syntdiag}
    `/'
    <pcre-definition>
    `/'
  \end{syntdiag}
\end{grammar}

Regular expression literals consist of a sequence of bytes, enclosed by
forward slash characters. The byte data should be understood as a PCRE,
with normal escaping rules.

\subsection{Data References}

Input data can be referenced by field index (one-based), or the entire
record.

\begin{grammar}
  <data> ::= \begin{syntdiag}
    \begin{stack}
      <column-id> \\
      <record>
    \end{stack}
  \end{syntdiag}

  <column-id> ::= \begin{syntdiag}
    `\$'
    \begin{rep}
      <DIGIT>
    \end{rep}
  \end{syntdiag}

  <record> ::= \begin{syntdiag}
    `\$0'
  \end{syntdiag}

  <column-ref> ::= \begin{syntdiag}
    `\%'
    \begin{rep}
      <DIGIT>
    \end{rep}
  \end{syntdiag}
\end{grammar}

Note that there must be no whitespace interposed within a column
identifier or reference.

\subsection{Expressions}

An expression represents anything that can be evaluated to a scalar
value, taking potentially scalar or vector input.

\begin{grammar}
  <expression> ::= \begin{syntdiag}
    \begin{stack}
      <expr-block> \\
      `(' <expr-block> `)'
    \end{stack}
  \end{syntdiag}

  <expr-block> ::= \begin{syntdiag}
    \begin{stack}
      <data> \\
      <data-literal> \\
      <expression> <operator> <expression> \\
      <function>
    \end{stack}
  \end{syntdiag}

  <operator> ::= \begin{syntdiag}
    \begin{stack}
      `+' \\
      `-' \\
      `*' \\
      `/' \\
      `\%' \\
      `^'
    \end{stack}
  \end{syntdiag}

  <function> ::= \begin{syntdiag}
    <symbol-literal>
    `('
    \begin{stack}
      <arg-list> \\
    \end{stack}
    `)'
  \end{syntdiag}

  <arg-list> ::= \begin{syntdiag}
    <expression>
    \begin{stack}
      \begin{rep}`,' <expression>\end{rep} \\
    \end{stack}
  \end{syntdiag}
\end{grammar}

Note that there must be no whitespace interposed between a function's
symbolic name and the opening parenthesis for its calling arguments.

\subsection{Output List}

The output list defines the columns in the final output; it must be
present in the \stag\ statement.

\begin{grammar}
  <out-list> ::= \begin{syntdiag}
    <out-column>
    \begin{stack}
      \begin{rep}`,' <out-column>\end{rep} \\
    \end{stack}
  \end{syntdiag}

  <out-column> ::= \begin{syntdiag}
    <expression>
    \begin{stack}
      `as' <string-literal> \\
    \end{stack}
  \end{syntdiag}
\end{grammar}

Note that the output list must contain at least one aggregate column
(i.e., using a vector function).

\subsection{\texttt{from} Clause}

The \texttt{from} clause defines the input source. If omitted, the input
defaults to \texttt{stdin}.

\begin{grammar}
  <from-clause> ::= \begin{syntdiag}
    `from'
    \begin{stack}
      <filepath> \\
      <fd>
    \end{stack}
  \end{syntdiag}

  <filepath> ::= \begin{syntdiag}
    <string-literal>
  \end{syntdiag}

  <fd> ::= \begin{syntdiag}
    `\&'
    \begin{rep}
      <DIGIT>
    \end{rep}
  \end{syntdiag}
\end{grammar}

Note that there must be no whitespace interposed within a file
descriptor.

\subsection{\texttt{split by} Clause}

The \texttt{split by} clause defines how input lines are to be split
into fields. If omitted, the line is split by horizontal tabs or
multiple whitespace (i.e.,
\texttt{/\textbackslash{}t|\textbackslash{}s\{2,\}/}).

\begin{grammar}
  <split-clause> ::= \begin{syntdiag}
    `split' `by' <regex-literal>
  \end{syntdiag}
\end{grammar}

\subsection{\texttt{when} Clause}

The \texttt{when} clause defines conditions that filter the input data.
There is no equivalent to SQL's \texttt{having} clause as \stag\ is
intended for continuous operation.\footnote{As a non-terminal element of
a Unix pipeline, presuming the input stream terminates, \texttt{awk}
could be used downstream to emulate the \texttt{having} clause.}

\begin{grammar}
  <when-clause> ::= \begin{syntdiag}
    `when'
    <condition>
  \end{syntdiag}

  <condition> ::= \begin{syntdiag}
    \begin{stack}
      <logic-block> \\
      `(' <logic-block> `)'
    \end{stack}
  \end{syntdiag}

  <logic-block> ::= \begin{syntdiag}
    <predicate>
    \begin{stack}
      \begin{rep}<junction> <condition>\end{rep} \\
    \end{stack}
  \end{syntdiag}

  <junction> ::= \begin{syntdiag}
    \begin{stack}
      `and' \\
      `or'
    \end{stack}
  \end{syntdiag}

  <predicate> ::= \begin{syntdiag}
    \begin{stack}
      `not' \\
    \end{stack}
    <expression>
    <test>
  \end{syntdiag}

  <test> ::= \begin{syntdiag}
    \begin{stack}
      \begin{stack}
        `=' \\
        `<' \\
        `>' \\
        `<=' \\
        `>=' \\
        `!='
      \end{stack} <expression> \\
      `in' `(' <arg-list> `)' \\
      `~=' <regex-literal>
    \end{stack}
  \end{syntdiag}
\end{grammar}

\subsection{\texttt{sort} Clause}

\begin{grammar}
  <sort-clause> ::= \begin{syntdiag}
    `sort' `on'
    <sort-column>
    \begin{stack}
      \begin{rep}`,' <sort-column>\end{rep} \\
    \end{stack}
  \end{syntdiag}

  <sort-column> ::= \begin{syntdiag}
    <column-ref>
    \begin{stack}
      `asc' \\
      `desc' \\
    \end{stack}
  \end{syntdiag}
\end{grammar}

If not specified, data will be presented on a first-come first-served
basis. If column sorting is specified with no ordering, then data will
be sorted in ascending order by default.

% TODO Collation rules?

\subsection{Extension Clause}

Additional scalar and vector functions to those defined herein may be
imported using the extension clause.

\begin{grammar}
  <extend-clause> ::= \begin{syntdiag}
    `using'
    \begin{rep}
      <filepath>
    \end{rep}
  \end{syntdiag}
\end{grammar}
