% Copyright (c) 2016 Genome Research Ltd.
% 
% Author: Christopher Harrison <ch12@sanger.ac.uk>
% 
% This file is part of stag.
% 
% stag is free software: you can redistribute it and/or modify it under
% the terms of the GNU General Public License as published by the Free
% Software Foundation, either version 3 of the License, or (at your
% option) any later version.
% 
% This program is distributed in the hope that it will be useful, but
% WITHOUT ANY WARRANTY; without even the implied warranty of
% MERCHANTABILITY or FITNESS FOR A PARTICULAR PURPOSE. See the GNU
% General Public License for more details.
% 
% You should have received a copy of the GNU General Public License
% along with this program. If not, see <http://www.gnu.org/licenses/>.

\section{Standard Functions}

We use the following markup for function signatures:

\begin{center}
\fn{function}{input type}{output type}
\end{center}

Where a union of types are accepted, these are delineated with
\texttt{|} characters. Conventionally, if the input and output types are
corresponding unions, it is expected that the output type matches the
input. Moreover, \texttt{Any} is used to denote the union of all types.

For aggregate functions, the input type (or union thereof) will be
enclosed in square brackets, to denote the vector input.

\subsection{Operators}

\paragraph{\fn{\^{}}{(Numeric, Numeric)}{Numeric}} Exponentiation
\paragraph{\fn{*}{(Numeric, Numeric)}{Numeric}} Multiplication
\paragraph{\fn{/}{(Numeric, Numeric)}{Numeric}} Division
\paragraph{\fn{\%}{(Numeric, Numeric)}{Numeric}} Modulus
\paragraph{\fn{+}{(Numeric, Numeric)}{Numeric}} Addition
\paragraph{\fn{-}{(Numeric, Numeric)}{Numeric}} Subtraction

\paragraph{\fn{+}{(Textual, Textual)}{Textual}} Concatenation

\subsection{Scalar Functions}

\subsubsection{Casting Functions}

Note that the implicit typing does away with most needs to cast.
However, if input data does not conform to \stag\ literals, then it
could be manipulated appropriately -- as a string -- and cast to the
correct type.

\paragraph{\fn{to_text}{Any}{Textual}}
\paragraph{\fn{to_numeric}{Textual}{Numeric}}
\paragraph{\fn{to_datetime}{Textual}{Temporal}}
\paragraph{\fn{to_regex}{Textual}{Regular Expression}}

\subsubsection{String Functions}

% TODO

\subsubsection{Datetime Functions}

% TODO

\subsection{Aggregate Functions}

\paragraph{\fn{count}{[Any]}{Numeric}}

The size of the input (i.e., number of elements).

\paragraph{\fn{sum}{[Numeric]}{Numeric}}

The sum of the input elements.

\paragraph{\fn{prod}{[Numeric]}{Numeric}}

The product of the input elements.

\paragraph{\fn{mean}{[Numeric]}{Numeric}}

The arithmetic mean of the input elements.

\paragraph{\fn{max}{[Numeric | Datetime]}{Numeric | Datetime}}

The maximum element from the input.

\paragraph{\fn{min}{[Numeric | Datetime]}{Numeric | Datetime}}

The minimum element from the input.

\paragraph{\fn{first}{[Any]}{Any}}

The first element from the input.

\paragraph{\fn{last}{[Any]}{Any}}

The latest element in the input.

% TODO? Would a tail function be more useful? i.e., Return all but the
% first element...

% TODO Others?
% Standard deviation (population and sample)
% Variance (population and sample)
% Median
% Arbitrary percentile
% ...
